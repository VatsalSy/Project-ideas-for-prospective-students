\documentclass[a4paper,10pt]{article}
\usepackage{fullpage}
\usepackage{float}
\usepackage[english]{babel}
\usepackage{graphicx,subfig,wrapfig}
\usepackage{amsmath,amsfonts,amsthm,amssymb}
\usepackage{fancyhdr,fancybox,color}
\usepackage{enumerate}
\usepackage[amssymb]{SIunits}
\definecolor{MyBlue}{rgb}{0,0.3,0.6}
\usepackage[colorlinks=true,
            linkcolor=MyBlue,
            plainpages=false,
            citecolor=MyBlue,
            urlcolor=MyBlue]{hyperref}
\usepackage[all]{hypcap}
\usepackage[url=false,
backend=bibtex,
style=authoryear-comp,
doi=true,
isbn=true,
backref=false,
dashed=false,
maxcitenames=2,
maxbibnames=99,
natbib=true]{biblatex}
\addbibresource{refrence.bib}
\nonfrenchspacing
\begin{document}
\noindent Chair: Physics of Fluids group
\begin{center}
 \begin{LARGE}
  Bubble in volcanoes -- Part 1: Experiments
 \end{LARGE}
\end{center}
\section*{Description}
Bubbles play an essential role in magmatic and mud volcanoes. It has been shown that the growth, motion, and burst of the bubbles at the interface of the conduit control the eruption (see Figure~\ref{Figure::Typical}). In this project, we aim to understand more about the role of bubbles in volcanoes. In particular, we will study the burst of bubbles at the liquid-gas interface that is believed to control the eruption.\\
In the lab, we will use a model fluid to mimic magma or mud, and we will generate the bubbles by injecting gas from the bottom of the container. We will study the burst in detail by high-speed imaging for different experimental conditions, such as different bubble size and different material properties.
\begin{figure}[H]
\begin{center}
 \includegraphics[width=\textwidth]{Fig1}
 \caption{(a) A schematic showing the role of bubbles in a volcanic eruption \citep{gonnermann2007fluid}. (b) A real-life image showing the hydrodynamics of volcanic eruption (google images).}
 \label{Figure::Typical}
\end{center}
\end{figure}
\section*{What you will do and what you will learn?}
In the Physics of Fluids group, we are looking for enthusiastic students to join our newly established projects on fluid mechanics of volcanoes.
\begin{enumerate}
\itemsep0em
\item You will learn about fundamental geophysical fluid mechanics and bubble dynamics.
\item You will learn how to design experiments from scratch.
\item You will learn how to use high-speed cameras and image processing.
\end{enumerate}
If you have any questions, fell free to contact \href{mailto:v.sanjay@utwente.nl}{Vatsal} or \href{mailto:m.jalaal@utwente.nl}{Mazi} (details below).
\begin{center}
\begin{tabular}{|l|l|l|}
\hline \textbf{Supervision} & \textbf{E-mail} & \textbf{Office} \\
\hline Vatsal Sanjay & \href{mailto:v.sanjay@utwente.nl}{v.sanjay@utwente.nl} & Meander 208 \\
\hline Dr. Maziyar (Mazi) Jalaal   & \href{mailto:m.jalaal@utwente.nl}{m.jalaal@utwente.nl}& Meander 208 \\
\hline Prof. D. Lohse & \href{mailto:d.lohse@utwente.nl}{d.lohse@utwente.nl} & Meander 261  \\
\hline
\end{tabular}
\end{center}
\printbibliography
\end{document}
